% \begin{figure}[ht!]
% 	\centering{
% 		\includegraphics[width=0.9\textwidth]{img/img1}
% 		\caption{Представление в памяти (A B C D).} }
% \end{figure}


\chapter{Теоретические сведения}

\textit{Моделирование} - исследование объектов, в ходе которого он заменяется моделью и исследование объекта проводится на его модели.
Результат переносится на исходный объект. 
Объектом может быть система, явление, процесс и т.д..

Модель -  представление объекта в виде, отличном от облика или способа его реального существования
или способа функционирования.

Корректно поставленная задача, если ее решение существует единственно и устойчиво по входным данным.

\textit{Устойчивая задача} - малое изменение входных данных должно порождать малые изменения выходных данных.

\section{ОДУ}

Дано ОДУ (Обыкновенное Дифференциальное уравнение) n-ого порядка (\ref{eq:ody}).

\begin{equation}
	F(x, u', u'', ... , u^{(n)} = 0)
	\label{eq:ody}
\end{equation}

ОДУ любого порядка может быть сведено к системе ОДУ 1-ого порядка.

\section{Задача Коши}

\textit{Задача Коши} состоит в нахождении решения дифференциального уравнения, удовлетворяющего начальным условиям. 
Это одна из основных задач теории дифференциальных уравнений.

Имеется задача Коши (\ref{eq:ref1}).

\begin{equation}
	{\begin{cases}
			u'(x) = f(x,u) \\
			u(\xi) = \eta
		\end{cases}}
		\label{eq:ref1}
\end{equation}

Методы решения ОДУ в задачи Коши:

\begin{enumerate}
	\item аналитические;
	\item приближенно аналитические;
	\item численные.
\end{enumerate}

\section{Методы решения задачи Коши}

При отсутствии аналитического решения можно воспользоваться приближенно аналитическим методом Пикара.
Заменив дифференциальное уравнение интегральным получим (\ref{eq:ref2}).

\begin{equation}
	y(x)^{s} = \eta + \int_{\xi}^{x} f(t, y^{s-1}(t)) dt
	\label{eq:ref2}
\end{equation}

\begin{equation}
y^{(0)} = \eta
\end{equation}

Метод сходится если:

\begin{enumerate}
	\item правая часть непрерывная;
	\item выполнено условие Липшица (\ref{eq:Lip})
\end{enumerate}

\begin{equation}
	|f(x, u_1) - f(x, u_2| \leq L |u_1-u_2|
	\label{eq:Lip}
\end{equation}
где L - константа Липшица.
	
Метод Эйлера (\ref{eq:ref3}).

\begin{equation}
	y_{n+1} = y_n + h*f(x_n, y_n)
	\label{eq:ref3}
\end{equation}


Метод Рунге-Кутты (\ref{eq:ref4}).

\begin{equation}
	y_{n+1} = y_n + h*[(1-\alpha)k_1 + \alpha * k_2]
	\label{eq:ref4}
\end{equation}

Где $k_1$ и $k_2$ представлены как (\ref{eq:ref5}) и (\ref{eq:ref6}) соответственно. А $\alpha$ = 1 или $\frac{1}{2}$

\begin{equation}
	k_1 = f(x_n, y_n)
	\label{eq:ref5}
\end{equation}

\begin{equation}
	k_2 = f(x_n + \frac{h}{2\alpha}, y_n + \frac{h}{2\alpha}k_1)
	\label{eq:ref6}
\end{equation}

\chapter{Задание и вычисления приближений для метода Пикара}

Дана задача {\ref{eq:task}}

\begin{equation}
	{\begin{cases}
			u'(x) = u^2 + x^2 \\
			u(0) = 0
		\end{cases}}
		\label{eq:task}
\end{equation}

Используя описанные выше методы построить таблицу для:

\begin{enumerate}
	\item метода Пикара:
	\begin{enumerate}
		\item Первое приближение;
		\item Второе приближение;
		\item третье приближение;
		\item четверное приближение.
	\end{enumerate}
	\item метод Эйлера;
	\item метод Рунге-Кутты.
\end{enumerate}

Приближения:

\begin{equation}
	y^{(1)} = 0 + \int_{0}^{x} t^2 dt= \frac{x^3}{3} 
\end{equation}

\begin{equation}
	y^{(2)} = 0 + \int_{0}^{x}\left[ \left( \frac{t^3}{3} \right)^2 + t^2 \right] dt = \frac{x^3}{3} + \frac{x^7}{63}
\end{equation}

\begin{equation}
	\begin{split}
		y^{(3)} = 0 + \int_{0}^{x}\left[ \left(\frac{t^7}{63} + \frac{t^3}{3} \right)^2 + t^2 \right] dt = \\ 
		\int_{0}^{x}\left[ \frac{t^{14}}{63^2} + \frac{2}{63*3}t^{10} + \frac{t^6}{9} + t^2 \right] = \\
		\frac{x^3}{3} + \frac{x^7}{63} + \frac{2x^{11}}{2079} + \frac{x^{15}}{59535}
	\end{split}
\end{equation}

\begin{equation}
	\begin{split}
		y^{(4)} = 0 + \int_{0}^{x}\left[ \left( \frac{t^3}{3} + \frac{t^7}{63} + \frac{2t^{11}}{2079} + \frac{t^{15}}{59535} \right)^2 + t^2 \right] dt = \\ 
		= \frac{x^3}{3} + \frac{x^{7}}{63} + \frac{2x^{11}}{2079} + \frac{13x^{15}}{218295} + \frac{82x^{19}}{37328445} + \\
		+ \frac{662x^{23}}{10438212015} + \frac{4x^{27}}{3341878155} + \frac{x^{31}}{109876903905}
	\end{split}
\end{equation}

\chapter{Листинги реализованных методов}

\begin{lstlisting}[language=Python]
from prettytable import PrettyTable

# Подбираем шаг:

# Euler:
# При 1e-1 y(1) = 0.2925421046
# При 1e-2 y(1) = 0.3331073593
# При 1e-3 y(1) = 0.3484859823
# При 1e-4 y(1) = 0.3501691515
# При 1e-5 y(1) = 0.3502255745
# Изменение шага ничего не меняет (между 1e-4 и 1e-5) 
# Значит мы подобрали нужный нам шаг.

# Runge:
# При 1e-1 y(1) = 0.3485453439
# При 1e-2 y(1) = 0.3391265967
# При 1e-3 y(1) = 0.3491103993
# При 1e-4 y(1) = 0.3502318426
# При 1e-5 y(1) = 0.3502318443
# Аналогично.

MAX_X = 1
STEP = 1e-4

	
def f(x, y):
	return pow(x, 2) + pow(y, 2)

def fp1(x):
	return pow(x, 3) / 3

def fp2(x):
	return pow(x, 7) / 63 + \
		fp1(x)

def fp3(x):
	return pow(x, 15) / 59535 + \
		2 * pow(x, 11) / 2079 + \
		fp2(x)

def fp4(x):
	return pow(x, 31) / 109876903905 + \
		4 * pow(x, 27) / 3341878155 + \
		662 * pow(x, 23) / 10438212015 + \
		82 * pow(x, 19) / 37328445 + \
		fp3(x)


def Picar(x_max, h, func):
	result = list()
	x, y = 0, 0

	while x < x_max:
		result.append(y)
		x += h
		y = func(x)
	
	return result


def Euler(x_max, h):
	result = list()
	x, y = 0, 0 	# Начальное условие.
	
	while x < x_max:
		result.append(y)
		y = y + h * f(x, y)
		x += h

	return result


def Runge(x_max, h):
	result = list()
	coeff = h / 2
	x, y = 0, 0
	
	while x < x_max:
		result.append(y)
		y = y + h * f(x + coeff, y + coeff * f(x, y))
		x += h
	
	return result


def x_range(x_max, h):
	result = list()
	x = 0
	while x < x_max:
		result.append(round(x, 2))
		x += h
	return result


def main():
	column_names = ["X","Picard 1", "Picard 2", "Picard 3", "Picard 4", "Runge"]

	tb = PrettyTable()
	tb.add_column("X", x_range(MAX_X, STEP))
	tb.add_column("Picard 1", Picar(MAX_X, STEP, fp1))
	tb.add_column("Picard 2", Picar(MAX_X, STEP, fp2))
	tb.add_column("Picard 3", Picar(MAX_X, STEP, fp3))
	tb.add_column("Picard 4", Picar(MAX_X, STEP, fp4))
	tb.add_column("Euler", Euler(MAX_X, STEP))
	tb.add_column("Runge", Runge(MAX_X, STEP))

	print(tb)


if __name__ == "__main__":
	main()	
\end{lstlisting}


\chapter{Ответы на вопросы}

% 1. Укажите интервалы значений аргумента, в которых можно считать решением заданного
% уравнения каждое из первых 4-х приближений Пикара. Точность результата оценивать до
% второй цифры после запятой. Объяснить свой ответ.

1. Для того, чтобы указать интервал значений аргумента, 
в которых можно считать решением заданного
уравнения каждое из первых 4-х приближений проанализируем
полученные значения.
Так как нам дано начальное приближение, то левой границей будет 0.
Для определения правой границы границы мы будем анализировать полученные
решения методом Пикара для конкретного приближения и сравнивать со значениями 
более высоких порядков приближения и с результатами численных методов при 
определенном шаге. В листинге был подобран шаг 1e-4.
Для первого приближения искомым интервалом будет [0, 0.89], 
для второго [0, 1.12], для третьего  [0, 1.39], для четвертого [0, 1.4]

% При данном шаге будут анализироваться значения.
% будем постепенно уменьшать шаг и сравнивать результаты для разных приближений.

2. В численных методах правильность полученного результата, при 
фиксированном значении аргумента, доказывается путем уменьшения шага.
Правильно полученный результат - это когда при уменьшение 
шага значение аргумента незначительно (или вообще) не меняется.

3. Примерное значение функции при x = 2.


\begin{equation}
	u(2) \approx 316.713
	\label{eq:ref4}ё
\end{equation}
