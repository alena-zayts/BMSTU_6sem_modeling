

\documentclass[12pt]{report}
\usepackage[utf8]{inputenc}
\usepackage[russian]{babel}
\usepackage[14pt]{extsizes}
\usepackage{listings}
\usepackage{graphicx}
\usepackage{amsmath,amsfonts,amssymb,amsthm,mathtools} 
\usepackage{pgfplots}
\usepackage{filecontents}
\usepackage{float}
\usepackage{indentfirst}
\usepackage{eucal}
\usepackage{enumitem}
%s\documentclass[openany]{book}
\frenchspacing

\usepackage{titlesec}
\titleformat{\section}
{\normalsize\bfseries}
{\thesection}
{1em}{}
\titlespacing*{\chapter}{0pt}{-30pt}{8pt}
\titlespacing*{\section}{\parindent}{*4}{*4}
\titlespacing*{\subsection}{\parindent}{*4}{*4}

\usepackage{indentfirst} % Красная строка

\usetikzlibrary{datavisualization}
\usetikzlibrary{datavisualization.formats.functions}

\usepackage{amsmath}


% Для листинга кода:
\lstset{ %
	language=c,                 % выбор языка для подсветки (здесь это С)
	basicstyle=\small\sffamily, % размер и начертание шрифта для подсветки кода
	numbers=left,               % где поставить нумерацию строк (слева\справа)
	numberstyle=\tiny,           % размер шрифта для номеров строк
	stepnumber=1,                   % размер шага между двумя номерами строк
	numbersep=5pt,                % как далеко отстоят номера строк от подсвечиваемого кода
	showspaces=false,            % показывать или нет пробелы специальными отступами
	showstringspaces=false,      % показывать или нет пробелы в строках
	showtabs=false,             % показывать или нет табуляцию в строках
	frame=single,              % рисовать рамку вокруг кода
	tabsize=2,                 % размер табуляции по умолчанию равен 2 пробелам
	captionpos=t,              % позиция заголовка вверху [t] или внизу [b] 
	breaklines=true,           % автоматически переносить строки (да\нет)
	breakatwhitespace=false, % переносить строки только если есть пробел
	escapeinside={\#*}{*)}   % если нужно добавить комментарии в коде
}


\usepackage[left=2cm,right=2cm, top=2cm,bottom=2cm,bindingoffset=0cm]{geometry}
% Для измененных титулов глав:
\usepackage{titlesec, blindtext, color} % подключаем нужные пакеты
\definecolor{gray75}{gray}{0.75} % определяем цвет
\newcommand{\hsp}{\hspace{20pt}} % длина линии в 20pt
% titleformat определяет стиль
\titleformat{\chapter}[hang]{\Huge\bfseries}{\thechapter\hsp\textcolor{gray75}{|}\hsp}{0pt}{\Huge\bfseries}


% plot
\usepackage{pgfplots}
\usepackage{filecontents}
\usetikzlibrary{datavisualization}
\usetikzlibrary{datavisualization.formats.functions}

\begin{document}
	%\def\chaptername{} % убирает "Глава"
	\thispagestyle{empty}
	\begin{titlepage}
		\noindent \begin{minipage}{0.15\textwidth}
			\includegraphics[width=\linewidth]{img/b_logo}
		\end{minipage}
		\noindent\begin{minipage}{0.9\textwidth}\centering
			\textbf{Министерство науки и высшего образования Российской Федерации}\\
			\textbf{Федеральное государственное бюджетное образовательное учреждение высшего образования}\\
			\textbf{~~~«Московский государственный технический университет имени Н.Э.~Баумана}\\
			\textbf{(национальный исследовательский университет)»}\\
			\textbf{(МГТУ им. Н.Э.~Баумана)}
		\end{minipage}
		
		\noindent\rule{18cm}{3pt}
		\newline\newline
		\noindent ФАКУЛЬТЕТ $\underline{\text{«Информатика и системы управления»}}$ \newline\newline
		\noindent КАФЕДРА $\underline{\text{«Программное обеспечение ЭВМ и информационные технологии»}}$\newline\newline\newline
		
		\begin{center}
			\noindent\begin{minipage}{1.1\textwidth}\centering
				\Large\textbf{  Отчет по лабораторной работе №1}\newline
				\textbf{по дисциплине <<Моделирование>>}\newline
			\end{minipage}
		\end{center}
		
		\noindent\textbf{Тема} $\underline{\text{Программная реализация приближенного аналитического метода и}}$\newline   
		$\underline{\text{		численных алгоритмов первого и второго порядков точности при}}$\newline
		$\underline{\text{решении задачи Коши для ОДУ}}$\newline\newline
		\noindent\textbf{Студент} $\underline{\text{Зайцева А. А.~~~~~~~~~~~~~~~~~~~~~~~~~~~~~~~~~~~~~~~~~~}}$\newline\newline
		\noindent\textbf{Группа} $\underline{\text{ИУ7-62Б~~~~~~~~~~~~~~~~~~~~~~~~~~~~~~~~~~~~~~~~~~~~~~~~~~}}$\newline\newline
		\noindent\textbf{Оценка (баллы)} $\underline{\text{~~~~~~~~~~~~~~~~~~~~~~~~~~~~~~~~~~~~~~~~~~~~~~~~~}}$\newline\newline
		\noindent\textbf{Преподаватель} $\underline{\text{Градов В. М.~~~~~~~~~~~~~~~~~~~~~~~~~~~~}}$\newline\newline\newline
		
		\begin{center}
			\vfill
			Москва~---~\the\year
			~г.
		\end{center}
	\end{titlepage}

\chapter{Задание}

\textbf{Цель работы.} 

Получение навыков решения задачи Коши для ОДУ методами Пикара и явными методами первого порядка точности (Эйлера) и второго порядка точности (Рунге-Кутты).

\textbf{Исходные данные.} 

ОДУ \ref{eq:task1}, не имеющее аналитического решения:

\begin{equation}
	{\begin{cases}
			u'(x) = u^2 + x^2 \\
			u(0) = 0.
	\end{cases}}
	\label{eq:task1}
\end{equation}

\textbf{Результат работы программы.}

1. Таблица, содержащая значения аргумента с заданным шагом в интервале [0, xmax] и результаты расчета функции u(x) в приближениях Пикара (от 1-го до 4-го), а также численными методами. Границу интервала xmax выбирать максимально возможной из условия, чтобы численные методы обеспечивали точность вычисления решения уравнения u(x) до второго знака после запятой.

\chapter{Теоретическая часть}

Обыкновенное дифференциальное уравнение (ОДУ) n-ого порядка имеет вид \ref{eq:ody}:

\begin{equation}
	F(x, u', u'', ... , u^{(n)} = 0)
	\label{eq:ody}.
\end{equation}

Задача Коши состоит в нахождении решения дифференциального уравнения, удовлетворяющего начальным условиям \ref{eq:ref1}:


\begin{equation}
	{\begin{cases}
			u'(x) = f(x,u) \\
			u(\xi) = \eta
	\end{cases}}
	\label{eq:ref1}
\end{equation}

Рассмотрим методы решения этой задачи.

%Методы решения ОДУ в задачи Коши:

%\begin{enumerate}
%	\item аналитические;
%	\item приближенно аналитические;
%	\item численные.
%\end{enumerate}

\section{Метод Пикара}

Метод Пикара является приближенно-аналитическим. Идея состоит в том, чтобы заменить дифференциальное уравнение интегральным \ref{eq:ref2}:.


\begin{equation}
	y^{s}(x) = \eta + \int_{\xi}^{x} f(t, y^{s-1}(t)) dt
	\label{eq:ref2}
\end{equation}

\begin{equation}
	y^{(0)} = \eta
\end{equation}

Метод сходится если правая часть непрерывна и выполнено условие Липшица: $|f(x, u_1) - f(x, u_2)| \leq L |u_1-u_2|$, где L - константа Липшица. 

Для данного в задании ОДУ \ref{eq:task1}:

\begin{equation}
	y^{(1)} = 0 + \int_{0}^{x} t^2 dt= \frac{x^3}{3} 
\end{equation}

\begin{equation}
	y^{(2)} = 0 + \int_{0}^{x}\left[ \left( t^2 + \frac{t^3}{3} \right)^2 \right] dt = \frac{x^3}{3} + \frac{x^7}{63}
\end{equation}

\begin{equation}
	\begin{split}
		y^{(3)} = 0 + \int_{0}^{x}\left[t^2 + \left(\frac{t^7}{63} + \frac{t^3}{3} \right)^2\right] dt = \\ 
		\frac{x^3}{3} + \frac{x^7}{63} + \frac{2x^{11}}{2079} + \frac{x^{15}}{59535}
	\end{split}
\end{equation}

\begin{equation}
	\begin{split}
		y^{(4)} = 0 + \int_{0}^{x}\left[t^2 + \left( \frac{t^3}{3} + \frac{t^7}{63} + \frac{2t^{11}}{2079} + \frac{t^{15}}{59535} \right)^2 \right] dt = \\ 
		= \frac{x^3}{3} + \frac{x^{7}}{63} + \frac{2x^{11}}{2079} + \frac{13x^{15}}{218295} + \frac{82x^{19}}{37328445} + \\
		+ \frac{662x^{23}}{10438212015} + \frac{4x^{27}}{3341878155} + \frac{x^{31}}{109876903905}
	\end{split}
\end{equation}

\section{Метод Эйлера}
Метод Эйлера -- это явный метод первого порядка точности, использующий формулу \ref{eq:ref3}:

\begin{equation}
	y_{n+1} = y_n + h*f(x_n, y_n).
	\label{eq:ref3}
\end{equation}


\section{Метод Рунге-Кутты}

Метод Ругнге-Кутты -- это явный метод второго порядка точности, использующий формулу \ref{eq:ref4}:

\begin{equation}
	y_{n+1} = y_n + h*[\frac{k_1}{2} + \frac{k_2}{2}],
	\label{eq:ref4}
\end{equation}
где $k_1 = f(x_n, y_n)$, $k_2 = f(x_n + h, y_n + hk_1)$.

%\begin{equation}
%	y_{n+1} = y_n + h*[(1-\alpha)k_1 + \alpha * k_2],
%	\label{eq:ref4}
%\end{equation}
%где $k_1 = f(x_n, y_n)$, $k_2 = f(x_n + \frac{h}{2\alpha}, y_n + \frac{h}{2\alpha}k_1)$, $\alpha$ = 1 или $\frac{1}{2}$
 
 исправить



Используя описанные выше методы построить таблицу для:

\begin{enumerate}
	\item метода Пикара:
	\begin{enumerate}
		\item Первое приближение;
		\item Второе приближение;
		\item третье приближение;
		\item четверное приближение.
	\end{enumerate}
	\item метод Эйлера;
	\item метод Рунге-Кутты.
\end{enumerate}




\chapter{Листинги реализованных методов}

\begin{lstlisting}[language=Python]
\end{lstlisting}


\chapter{Ответы на вопросы}

% 1. Укажите интервалы значений аргумента, в которых можно считать решением заданного
% уравнения каждое из первых 4-х приближений Пикара. Точность результата оценивать до
% второй цифры после запятой. Объяснить свой ответ.

1. Для того, чтобы указать интервал значений аргумента, 
в которых можно считать решением заданного
уравнения каждое из первых 4-х приближений проанализируем
полученные значения.
Так как нам дано начальное приближение, то левой границей будет 0.
Для определения правой границы границы мы будем анализировать полученные
решения методом Пикара для конкретного приближения и сравнивать со значениями 
более высоких порядков приближения и с результатами численных методов при 
определенном шаге. В листинге был подобран шаг 1e-4.
Для первого приближения искомым интервалом будет [0, 0.89], 
для второго [0, 1.12], для третьего  [0, 1.39], для четвертого [0, 1.4]

% При данном шаге будут анализироваться значения.
% будем постепенно уменьшать шаг и сравнивать результаты для разных приближений.

2. В численных методах правильность полученного результата, при 
фиксированном значении аргумента, доказывается путем уменьшения шага.
Правильно полученный результат - это когда при уменьшение 
шага значение аргумента незначительно (или вообще) не меняется.

3. Примерное значение функции при x = 2.


\begin{equation}
	u(2) \approx 316.713
	\label{eq:ref4}ё
\end{equation}


	\bibliographystyle{utf8gost705u}  % стилевой файл для оформления по ГОСТу
	
	\bibliography{51-biblio}          % имя библиографической базы (bib-файла)
	
	
\end{document}
